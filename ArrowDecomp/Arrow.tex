\documentclass{article}
\usepackage[a4paper, margin=2.5cm]{geometry}
\usepackage{amsmath}
\usepackage{caption}
\usepackage{placeins}
\usepackage{graphicx}
\usepackage{subcaption}
\usepackage{setspace}
\usepackage{float}

%\usepackage[active,tightpage]{preview}
\usepackage{natbib}
\bibpunct{(}{)}{,}{a}{}{;} 
\usepackage{url}
\usepackage{nth}
\usepackage{authblk}
% for the d in integrals
\newcommand{\dd}{\; \mathrm{d}}
\newcommand{\tc}{\quad\quad\text{,}}
\newcommand{\tp}{\quad\quad\text{.}}
\defcitealias{HMD}{HMD}

\newcommand\ackn[1]{%
  \begingroup
  \renewcommand\thefootnote{}\footnote{#1}%
  \addtocounter{footnote}{-1}%
  \endgroup
}
\begin{document}

%\title{Macro patterns in the shape of aging}
\title{Flow decompositions in multistate Markov models}
\author[1]{Tim Riffe\thanks{riffe@demogr.mpg.de}}
\affil[1]{Max Planck Institute for Demographic Research}
\maketitle

\begin{abstract}
We demonstrate the application of standard decomposition techniques to decompose
differences between synthetic indices derived from age-stage Markov matrix
models into differences due to each stage transition. An example is given on the basis
of transition matrices from an analysis of working life expectancy in the
United States.
\end{abstract}

\section{Introduction}
I describe the application of a generic pseudo-continuous time decomposition
method \citep{horiuchi2008} differences between synthetic indices derived
from two sets of transition probabilities into differences from each each
age-stage transition (in our case aggregated over age). Intuitively this means
we can assign how much of a difference is due to differences in each arrow in
the state-space diagram of the model in question. We demonstrate this
decomposition technique using published transition matrices from a recent study
of working life expectancy in the United States \citep{Dudel2017}.

\singlespacing
\bibliographystyle{plainnat}
  \bibliography{references} 

\end{document}
